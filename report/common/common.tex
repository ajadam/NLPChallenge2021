\documentclass[11.5pt, reqno, oneside]{amsbook}
%\documentclass{tufte-book}

% Utility packages
\usepackage{etoolbox}
\newtoggle{tufte}
%\toggletrue{tufte}
\usepackage{amsthm}
\usepackage{amsmath}
\usepackage{amssymb}
\usepackage{geometry}
\usepackage{setspace}
\usepackage{graphicx}
\usepackage{enumerate}
\usepackage{listings}
\usepackage{booktabs}
\usepackage{hyperref}
\usepackage{enotez}
\usepackage{thmtools}
\usepackage{xparse}
\usepackage{xspace}
\usepackage{enumitem}
\usepackage{todonotes}
\usepackage{dcolumn}
\usepackage{array}
\usepackage{booktabs}
\usepackage{colortbl, xcolor}
\usepackage{longtable}
\usepackage{natbib}
\usepackage{python}
\usepackage{bbm}
\usepackage[ruled,vlined]{algorithm2e}
\usepackage{mathtools}
\usepackage{bm}

% Images path
\graphicspath{ {./images/} }

\hypersetup{colorlinks=true}

% Custom logic for Tufte-LaTeX vs AMSBook
\iftoggle{tufte}{
  % TUFTE
  \publisher{Master MAS\\Université Rennes 2}
}{
  % AMSBOOK
  \newcommand{\sidenote}{\footnote}
  \numberwithin{equation}{chapter}
}

\declaretheorem[numberwithin=chapter,name=Theorem]{thm}
\declaretheorem[name=Lemma,sibling=thm]{lem}
\declaretheorem[name=Corollary,sibling=thm]{corollary}
\declaretheorem[name=Definition,sibling=thm,style=definition]{defn}
\declaretheorem[name=Proposition,sibling=thm]{proposition}
\declaretheorem[name=Example,sibling=thm]{exmp}
\declaretheorem[name=Exercise,sibling=thm]{exer}
\declaretheorem[name=History,sibling=thm]{history}
\declaretheorem[name=Question,sibling=thm]{question}
\declaretheorem[name=Remark,sibling=thm]{remark}
\declaretheorem[name=Notation,numbered=no]{notation}
\declaretheorem[name=Theorem,sibling=thm,thmbox=L]{boxthm}

\let\proof\relax
\declaretheorem[style=remark,numbered=no,qed=\qedsymbol]{proof}

\DeclareMathOperator*{\argmax}{arg\,max}
\DeclareMathOperator*{\argmin}{arg\,min}
\DeclareMathOperator*{\aff}{aff}
\DeclareMathOperator*{\rint}{rint}
\DeclareMathOperator*{\lev}{lev}
\DeclareMathOperator*{\tr}{tr}
\DeclareMathOperator*{\interior}{int}
\DeclareMathOperator*{\epi}{epi}
\DeclareMathOperator*{\dom}{dom}
\DeclareMathOperator*{\vecspan}{span}
\DeclareMathOperator*{\con}{con}
\DeclareMathOperator*{\cl}{cl}
\newcommand{\grad}{\nabla}
\DeclareMathOperator*{\crit}{crit}
\DeclareMathOperator*{\diag}{diag}
\DeclareMathOperator*{\res}{Res}
\DeclareMathOperator*{\boundary}{boundary}
\DeclareMathOperator*{\range}{range}
\DeclareMathOperator*{\Tr}{Tr}
\DeclareMathOperator*{\sign}{sign}


\setlist[enumerate]{label=(\roman*)}

\newlist{exercises}{enumerate}{1}
\setlist[exercises]{label=Ex. \arabic*}


\setcounter{secnumdepth}{2}
\setcounter{tocdepth}{2}
\onehalfspacing

\author{Loïc Rakotoson, Dylan Monfret, Antoine Adam - WeTried}


% Shortcuts
% convergence in probability
\newcommand{\cd}{\overset{d}{\rightarrow}}
% convergence in distribution
\newcommand{\cp}{\overset{p}{\rightarrow}}
% convergence almost surely
\newcommand{\cas}{\overset{as}{\rightarrow}}
\newcommand{\R}{\mathbb{R}}
\newcommand{\N}{\mathbb{N}}
\newcommand{\Z}{\mathbb{Z}}
\newcommand{\np}{\textsc{np}\xspace}
\newcommand{\iid}{\xspace\textsc{iid}\xspace}
\newcommand{\knn}{$k$-\textsc{nn}\xspace}
\newcommand{\ifonlyif}{\Longleftrightarrow}

% Expectation
\NewDocumentCommand{\E}{gg}{%
  \IfNoValueTF{#1}
    {%
      \mathbb{E}
    }
    {%
      \IfNoValueTF{#2}
      {%
        \mathbb{E}\!\left(#1\right)
      }
      {%
        \mathbb{E}_{#2}\!\left(#1\right)
      }
    }%
}

% Probability with measure P
\NewDocumentCommand{\Prob}{gg}{%
  \IfNoValueTF{#1}
    {%
      \mathbb{P}
    }
    {%
      \IfNoValueTF{#2}
      {%
        \mathbb{P}\!\left(#1\right)
      }
      {%
        \mathbb{P}_{#2}\!\left(#1\right)
      }
    }%
}

% Probability with measure Q
\NewDocumentCommand{\Q}{g}{%
  \IfNoValueTF{#1}
    {%
      \mathbb{Q}
    }
    {%
      \mathbb{Q}\!\left(#1\right)
    }%
}


% Indicator
\NewDocumentCommand{\1}{g}{%
  \IfNoValueTF{#1}
    {%
      \mathbbm{1}
    }
    {%
      \mathbbm{1}\!\left(#1\right)
    }%
}


\NewDocumentCommand{\Var}{g}{%
  \IfNoValueTF{#1}
    {%
      \mathbb{V}
    }
    {%
      \mathbb{V}\!\left(#1\right)
    }%
}


\NewDocumentCommand{\Cov}{gg}{%
  \IfNoValueTF{#1}
    {%
      \text{Cov}
    }
    {%
      \text{Cov}\!\left(#1, #2\right)
    }%
}

% Inner product
\NewDocumentCommand{\IP}{g}{%
    {%
      \left \langle #1 \right \rangle
    }%
  }

  % Indicator
\NewDocumentCommand{\upperbound}{g}{%
  \IfNoValueTF{#1}
    {%
      \lceil \rceil
    }
    {%
      \lceil#1\rceil
    }%
}