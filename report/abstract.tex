\chapter{Résumé}

Dans ce rapport, nous présentons nos essais et résultats pour le Défi IA 2021 organisé par l'INSA Toulouse. Le problème consiste en une classification de textes parmi 28 classes, les classes étant des métiers. Les solutions sont jugées selon un critère de performance par rapport à la classification (le F1-score) mais aussi par rapport à la fairness, mesuré par le disparate impact qui vise à mesurer à quel point une solution est biaisée vis à vis du genre. \\
Nous exposons tout d'abord la problématique du concours, ainsi que les métriques sur lesquelles nous sommes jugées, puis passons à la description de toutes les méthodes que nous avons essayé. Nous commençons bien entendu par des méthodes statistiques classiques, après avoir  vectorisé nos textes (que ce soit via une procédure tf-idf, ou un simple de comptage des mots). Le résultat en terme de score se situe à 0.75 en utilisant une machine à support vecteur. Cependant un problème apparaît dans le fait que certaines classes soit beaucoup moins représentées que d'autres, et que nous arrivons moins bien à les prédire. Nous commençons donc à ce moment à nous intéresser à des méthodes de rééquilibrage des données comme SMOTE. Cependant, dans le même temps, nous commençons à tester des méthodes de transfer-learning via les architectures transformers, les résultats sont tout de suite meilleurs, mais plus le modèle pré-entrainé devient important en terme de taille, plus le résultat est instable. C'est pourquoi, après nous être renseigné à propos de ce problème, que nous choisissons de faire un ensemble de plusieurs prédictions, à travers un soft-voting classifier. Ces prédictions sont obtenu avec le même modèle pré-entrainé (RoBERTa large) que nous fine-tunons plusieurs fois avec différentes graines aléatoires. Enfin, nous essayons d'améliorer le disparate impact via  RoBERTa NLI, car ce modèle était celui qui avait le disparate impact le plus bas lors de nos tests (tout en baissant un petit peu le F1-score). Malheureusement, lors de la soumission, l'effet inverse ce produit.

%%% Local Variables: 
%%% mode: latex
%%% TeX-master: master
%%% End: 